%─────────────────────────────────────────────────────────────────────────────
% Kapitel 1: Einleitung
%─────────────────────────────────────────────────────────────────────────────
\chapter{Einleitung}
\label{chap:einleitung}

Deepfakes sind synthetisch erzeugte Medien, bei denen das Gesicht einer Person realistisch in Video- oder Bildmaterial eingefügt wird. Mit wachsender Rechenleistung und verbesserten Algorithmen lassen sich derartige Fälschungen immer schwieriger mit bloßem Auge erkennen. Die damit einhergehenden Risiken reichen von politischer Manipulation bis hin zum Identitätsdiebstahl. Ziel dieser Arbeit ist es, auf Basis moderner Deep-Learning‐Techniken automatische Verfahren zur Erkennung von Deepfakes zu entwickeln und zu evaluieren.

In den letzten Jahren wurden zahlreiche Methoden vorgestellt, die tiefe neuronale Netze nutzen, um Artefakte in Deepfake-Bildern zu detektieren (vgl. \cite{agarwal2020detecting}, \cite{rossler2019faceforensics++}). Viele Arbeiten beschränken sich auf bestimmte Arten von Deepfake‐Algorithmen (z. B. GAN‐basiert), während andere einen breiteren Ansatz wählen, der auf Merkmale wie unnatürliche Augenreflexion, fehlende Gesichtsmerkmale oder Inkonsistenzen in den Spatial-Frequenzen abzielt. 

In Kapitel \ref{chap:datensaetze} werden die verwendeten Datensätze genauer beschrieben. Kapitel \ref{chap:methodik} stellt die grundlegende Methodik vor, gefolgt von technischen Details zur Implementierung in Kapitel \ref{chap:implementierung}. Die experimentelle Evaluation und die erzielten Ergebnisse sind in Kapitel \ref{chap:evaluation} zusammengefasst. Abschließend erfolgt in Kapitel \ref{chap:diskussion} eine kritische Reflexion der gefundenen Resultate, bevor in Kapitel \ref{chap:fazit} ein Fazit gezogen wird.
