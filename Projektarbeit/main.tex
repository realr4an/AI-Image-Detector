%─────────────────────────────────────────────────────────────────────────────
% main.tex
% Wissenschaftliche Projektarbeit (ca. 30 Seiten) mit modularen Kapiteln
%─────────────────────────────────────────────────────────────────────────────
\documentclass[
    paper=a4,                % A4-Papier
    fontsize=11pt,           % 11pt Schriftgröße
    headings=normal,         % Normale Kapitälchen-Überschriften
    parskip=half,            % Halbzeiliger Abstand zwischen Absätzen
    listof=totoc,            % Abbildungsverzeichnis ins Inhaltsverzeichnis
    bibliography=totoc,      % Literaturverzeichnis ins Inhaltsverzeichnis
    toc=bibliography,        % Inhaltsverzeichnis mit Bibliographie-Nummer
    chapterprefix=false      % „Kapitel“-Präfix weglassen, falls gewünscht
]{scrreprt}

% ─────────────────────────────────────────────────────────────────────────────
% Paket‐Ladebefehle und Grundeinstellungen
% ─────────────────────────────────────────────────────────────────────────────
\usepackage[T1]{fontenc}              % T1 Kodierung (Trennung von Umlauten)
\usepackage[utf8]{inputenc}           % UTF-8 Eingabekodierung
\usepackage[ngerman]{babel}           % Deutsche Dokumentensprache
\usepackage{csquotes}                 % Korrekte Anführungszeichen, benötigt von biblatex
\usepackage{microtype}                % Verbesserte typografische Feineinstellungen
\usepackage[hyphens]{url}


% Abstand & Layout
\usepackage{setspace}                 % Zeilenabstand: \onehalfspacing, \doublespacing
\onehalfspacing                       % 1,5-zeiliger Abstand (De-facto Standard)
\usepackage{geometry}                 
\geometry{
  a4paper,
  left=30mm,
  right=25mm,
  top=25mm,
  bottom=25mm
}

% Kopf-/Fußzeilen
\usepackage[automark]{scrlayer-scrpage}
\clearpairofpagestyles
\ihead{\headmark}                     % Kapitelüberschrift in den Kopf
\ofoot{\pagemark}                     % Seitennummer im Fuß
\setkomafont{pageheadfoot}{\small}    % kleine Schrift in Kopf/Fuß

% Grafiken und Tabellen
\usepackage{graphicx}                 
\usepackage{caption}                  
\usepackage{subcaption}               % Unterabbildungen
\usepackage{booktabs}                 % Professionelle Tabellenlinien
\usepackage{tabularx}                 % Erweiterte Tabellenspalten
\usepackage{longtable}                % Tabellen, die sich über Seiten erstrecken

% Mathematik
\usepackage{amsmath,amssymb,amsthm}    % Mathe‐Umgebungen und Symbole

% Farben, Quellcode, Hyperlinks
\usepackage{xcolor}                   
\usepackage{listings}                 % Quellcode-Auszeichnung
\usepackage{hyperref}                 
\hypersetup{
    colorlinks=true,
    linkcolor=blue,
    citecolor=blue,
    urlcolor=blue
}
\usepackage{cleveref}                 % Intelligente Verweise: \cref, \Cref

% Bibliographie (BibLaTeX)
\usepackage[
    backend=biber,                    % biber als Backend
    style=alphabetic,                 % z.B. alphabetischer Stil (kann angepasst werden)
    citestyle=authoryear,             % Zitierstil: (Autor Jahr)
    sorting=nyt,                      % Sortierung: Name, Jahr, Titel
]{biblatex}
\addbibresource{literature.bib}       % Deine BibTeX-Datei

% Eigene Makros, Definitionen (falls benötigt)
\newcommand{\HRule}{\rule{\linewidth}{0.5mm}}   % Horizontaler Strich
% … hier kannst du weitere Makros definieren

% ─────────────────────────────────────────────────────────────────────────────
% Beginn des Dokuments
% ─────────────────────────────────────────────────────────────────────────────
\begin{document}

% Titelseite
%─────────────────────────────────────────────────────────────────────────────
% Titelseite deiner Projektarbeit
%─────────────────────────────────────────────────────────────────────────────
\thispagestyle{empty}
\begin{titlepage}
    \begin{center}
        \vspace*{2cm}
        {\Huge\textbf{Erkennung von Deepfakes\\ mittels Convolutional Neural Networks}}\\
        \vspace{1cm}
        {\Large Projektarbeit im Studiengang Informatik}\\
        \vspace{2cm}

        {\Large \textbf{Vorname Nachname}}\\
        \vspace{0.5cm}
        Matrikelnummer: 1234567\\
        \vspace{1cm}

       % \includegraphics[width=0.4\textwidth]{figures/universitaet-logo.png}\\   % ggf. Logo einfügen
        \vspace{1cm}

        {\large Abgabedatum: 05.\ Juni 2025}\\
        \vfill

        Betreuer: Prof. Dr. Max Mustermann\\
        Zweitkoordinator: Dr. Erika Beispiel
    \end{center}
\end{titlepage}


% Abstract (Zusammenfassung)
%─────────────────────────────────────────────────────────────────────────────
% Kurze Zusammenfassung (Abstract) auf Deutsch, 150–250 Wörter
%─────────────────────────────────────────────────────────────────────────────
\chapter*{Abstract}
\addcontentsline{toc}{chapter}{Abstract}

Im Rahmen dieser Projektarbeit wird untersucht, wie Convolutional Neural Networks (CNNs) eingesetzt werden können, um Deepfake-Bilder von echten Aufnahmen zu unterscheiden. Dazu werden verschiedene Datensätze (u. a. Kaggle–Deepfake‐Datensätze) zusammengeführt, vorverarbeitet und als Trainings- und Validierungssets aufbereitet. Als Basis dient ein vortrainiertes ResNet50‐Netzwerk, das durch Transfer Learning auf die Erkennung von Deepfakes adaptiert wird. Der Trainingsprozess umfasst sowohl das Einfrieren der Basislayer als auch ein anschließendes Fine-Tuning. Die Ergebnisse werden anhand von Kennzahlen wie Accuracy, Precision, Recall, F1-Score und AUC evaluiert. Abschließend werden Limitationen und mögliche Verbesserungsstrategien diskutiert. 


% Inhaltsverzeichnis
\clearpage
\tableofcontents
\clearpage

% ─────────────────────────────────────────────────────────────────────────────
% 1. Einleitung
% ─────────────────────────────────────────────────────────────────────────────
%─────────────────────────────────────────────────────────────────────────────
% Kapitel 1: Einleitung
%─────────────────────────────────────────────────────────────────────────────
\chapter{Einleitung}
\label{chap:einleitung}

Deepfakes sind synthetisch erzeugte Medien, bei denen das Gesicht einer Person realistisch in Video- oder Bildmaterial eingefügt wird. Mit wachsender Rechenleistung und verbesserten Algorithmen lassen sich derartige Fälschungen immer schwieriger mit bloßem Auge erkennen. Die damit einhergehenden Risiken reichen von politischer Manipulation bis hin zum Identitätsdiebstahl. Ziel dieser Arbeit ist es, auf Basis moderner Deep-Learning‐Techniken automatische Verfahren zur Erkennung von Deepfakes zu entwickeln und zu evaluieren.

In den letzten Jahren wurden zahlreiche Methoden vorgestellt, die tiefe neuronale Netze nutzen, um Artefakte in Deepfake-Bildern zu detektieren (vgl. \cite{agarwal2020detecting}, \cite{rossler2019faceforensics++}). Viele Arbeiten beschränken sich auf bestimmte Arten von Deepfake‐Algorithmen (z. B. GAN‐basiert), während andere einen breiteren Ansatz wählen, der auf Merkmale wie unnatürliche Augenreflexion, fehlende Gesichtsmerkmale oder Inkonsistenzen in den Spatial-Frequenzen abzielt. 

In Kapitel \ref{chap:datensaetze} werden die verwendeten Datensätze genauer beschrieben. Kapitel \ref{chap:methodik} stellt die grundlegende Methodik vor, gefolgt von technischen Details zur Implementierung in Kapitel \ref{chap:implementierung}. Die experimentelle Evaluation und die erzielten Ergebnisse sind in Kapitel \ref{chap:evaluation} zusammengefasst. Abschließend erfolgt in Kapitel \ref{chap:diskussion} eine kritische Reflexion der gefundenen Resultate, bevor in Kapitel \ref{chap:fazit} ein Fazit gezogen wird.


% ─────────────────────────────────────────────────────────────────────────────
% 2. Datensätze
% ─────────────────────────────────────────────────────────────────────────────
%─────────────────────────────────────────────────────────────────────────────
% Kapitel 2: Datensätze
%─────────────────────────────────────────────────────────────────────────────
\chapter{Datensätze}
\label{chap:datensaetze}

Ein zentraler Baustein dieser Arbeit war die sorgfältige Auswahl und systematische Vorverarbeitung verschiedener Deepfake-Bilddatensätze, um eine robuste und realitätsnahe Klassifikation zu ermöglichen. Für das Training und die Evaluierung des Modells wurden insgesamt drei öffentliche Datensätze von Kaggle herangezogen, welche im Folgenden näher beschrieben werden.

\section{Verwendete Datensätze}

\begin{itemize}
    \item \textbf{Deepfake-vs-Real-Classification}\footnote{\url{https://www.kaggle.com/datasets/prithivsakthiur/deepfake-vs-real-60k}}:  
    Dieser Datensatz enthält etwa 28.600 echte (\emph{real}) und 28.600 Deepfake-Bilder (\emph{fake}), sodass beide Klassen annähernd gleich stark vertreten sind. Die Bilder liegen im JPG-Format vor und zeigen unterschiedlichste Gesichter in diversen Umgebungen.

    \item \textbf{Detect AI-Generated Faces: High-Quality Dataset}\footnote{\url{https://www.kaggle.com/datasets/shahzaibshazoo/detect-ai-generated-faces-high-quality-dataset}}:  
    Ein kleiner, aber qualitativ besonders hochwertiger Datensatz, der speziell für die Unterscheidung von KI-generierten Gesichtern entwickelt wurde. Er enthält 1.001 Fake-Bilder und 2.002 Real-Bilder, jeweils mit klarer Zuordnung.

    \item \textbf{deepfake and real images}\footnote{\url{https://www.kaggle.com/datasets/manjilkarki/deepfake-and-real-images}}:  
    Der größte im Projekt verwendete Datensatz umfasst rund 95.092 als \emph{fake} und 95.213 als \emph{real} gelabelte Bilder, insgesamt also etwa 190.000 Gesichter. Die Bilder sind bereits im Standardformat (256x256 Pixel, JPG) verfügbar.
\end{itemize}

Alle Datensätze wurden zunächst unabhängig voneinander gesichtet, geprüft und nach den beiden Zielklassen (\emph{real}, \emph{fake}) kategorisiert.

\section{Datenorganisation und -aufteilung}

Um eine einheitliche und automatisiert verarbeitbare Datenbasis für das Deep-Learning-Modell zu schaffen, wurden sämtliche Bilder zunächst in eine gemeinsame Verzeichnisstruktur überführt:

\begin{verbatim}
Dataset/
|-- train/
|   |-- real/
|   \-- fake/
|-- validation/
|   |-- real/
|   \-- fake/
\-- test/
    |-- real/
    \-- fake/
\end{verbatim}

Die Bilder wurden anhand ihrer Ursprungsdatensätze und Labels den jeweiligen Unterordnern \texttt{real} und \texttt{fake} zugeordnet und dabei auf die drei Datensplits Training, Validierung und Test verteilt.

\section{Vorverarbeitung und Datenpipeline}

Die technische Vorverarbeitung der Bilder erfolgt im Training vollständig automatisiert durch den Einsatz eines \texttt{ImageDataGenerator} aus Keras, kombiniert mit der spezifischen Vorverarbeitungsfunktion für ResNet50 (\texttt{preprocess\_input}). Das Vorgehen ist wie folgt:

\begin{itemize}
    \item \textbf{Bildgröße:}  
    Alle Bilder werden beim Laden auf eine einheitliche Auflösung von 256x256 Pixeln gebracht (per \texttt{target\_size=(256, 256)}), um optimal mit der Architektur von ResNet50 zu harmonieren.

    \item \textbf{Normalisierung:}  
    Die Pixelwerte werden nicht nur skaliert, sondern gemäß der ResNet50-Konvention kanalweise normalisiert, um einen stabilen Wertebereich für das Netzwerk zu gewährleisten.

    \item \textbf{Datenaugmentation:}  
    Während zu Projektbeginn kurzzeitig verschiedene Augmentationstechniken (z.B. Rotation, Flip, Zoom) getestet wurden, verzichtet der finale Stand vollständig auf künstliche Erweiterungen oder Verzerrungen. Die Trainingsdaten werden also ohne zusätzliche Transformationen verwendet.

    \item \textbf{Batching und Label-Encoding:}  
    Die Bilddaten werden im Trainingsprozess in Batches zu je 64 Bildern geladen. Die Zuordnung zu den Klassen (\emph{real} oder \emph{fake}) erfolgt automatisch anhand der Verzeichnisstruktur und wird als binäre Label (0/1) dem Modell zugeführt.
\end{itemize}

Die Verwendung der \texttt{flow\_from\_directory}-Funktion gewährleistet, dass alle Splits (Training, Validation, Test) effizient und speicherschonend verarbeitet werden können.

\section{Zusammenfassung der Datenbasis}

Zusammengefasst standen nach Abschluss der Vorverarbeitung und Sortierung folgende Datenmengen für das Projekt zur Verfügung:

\begin{itemize}
    \item \textbf{Deepfake-vs-Real-Classification:} ca. 28.600 Real, 28.600 Fake
    \item \textbf{Detect AI-Generated Faces:} 2.002 Real, 1.001 Fake
    \item \textbf{deepfake and real images:} ca. 95.213 Real, 95.092 Fake
\end{itemize}

Durch die einheitliche Struktur und konsequente Normalisierung wurden optimale Voraussetzungen geschaffen, um ein robustes Deep-Learning-Modell auf Basis großer und vielseitiger Bilddaten zu trainieren. Auf manuelle oder künstliche Balancierung der Klassen wurde bewusst verzichtet, da die verwendeten Datensätze bereits eine weitgehende Gleichverteilung von echten und gefälschten Bildern aufweisen.


% ─────────────────────────────────────────────────────────────────────────────
% 3. Methodik
% ─────────────────────────────────────────────────────────────────────────────
%─────────────────────────────────────────────────────────────────────────────
% Kapitel 3: Methodik
%─────────────────────────────────────────────────────────────────────────────
\chapter{Methodik}
\label{chap:methodik}

\section{Experimentumgebung}

Die Durchführung sämtlicher Trainings- und Evaluationsläufe erfolgte auf einer lokalen Workstation mit NVIDIA RTX 2070 Super GPU (8~GB VRAM) unter Windows 10. Für das gesamte Deep-Learning-Framework kam TensorFlow (Version 2.x) in Verbindung mit Keras zum Einsatz. Der Trainingsprozess wurde über Visual Studio Code gesteuert und sowohl GPU- als auch CPU-Implementierungen wurden automatisch erkannt und genutzt. Insbesondere die Aktivierung des Mixed Precision Trainings (\texttt{mixed\_float16}) trug dazu bei, die GPU-Auslastung zu optimieren und die Speichereffizienz sowie Trainingsgeschwindigkeit deutlich zu steigern. Typischerweise betrug die Dauer eines Trainingsschritts (bei einer Batchgröße von 64) im Mittel ca.~385~ms.

\section{Verwendete Datensätze}

Für das Training und die Validierung des Deepfake-Klassifikators wurden mehrere öffentlich verfügbare Bilddatensätze von Kaggle integriert (siehe Kapitel~\ref{chap:datensaetze}). Im Fokus standen:

\begin{itemize}
  \item \textbf{Deepfake and Real Images:}  
    Hauptdatensatz mit 190.000 Bildern (256$\times$256~px, JPG), aufgeteilt in 140.000 Trainingsbilder, 39.400 Validierungsbilder und 10.905 Testbilder.
  \item \textbf{Detect AI-Generated Faces (High Quality):}  
    Ergänzend hinzugefügt, um die Varianz im Training zu erhöhen, mit 1.001 Fake- und 2.202 Real-Bildern.
\end{itemize}

Alle Daten wurden auf einen konsistenten Verzeichnisbaum (\texttt{train/}, \texttt{validation/}, \texttt{test/}, jeweils mit Unterordnern \texttt{real}/\texttt{fake}) gebracht, um eine automatisierte, fehlerfreie Datenzufuhr während des Trainings zu gewährleisten.

\section{Modellarchitektur}

Das eingesetzte neuronale Netz basiert auf einem vortrainierten \texttt{ResNet50}-Backbone (ImageNet-Gewichte, ohne Top-Layer). Diese Architektur wurde gewählt, da sie in der Literatur als sehr robust für Bildklassifikationsaufgaben gilt und in eigenen Vorversuchen anderen Modellen wie EfficientNetB0 überlegen war.

Der eigentliche Klassifikationskopf besteht aus den folgenden Schichten:

\begin{itemize}
  \item \texttt{GlobalAveragePooling2D()} als Übergang von Feature Maps zu einem Vektor.
  \item \texttt{Dense(128)}-Schicht mit ReLU-Aktivierung und L2-Regularisierung ($\lambda=0{,}002$), um Overfitting zu begegnen.
  \item \texttt{Dropout(0{,}6)} zur weiteren Reduzierung von Überanpassung.
  \item \texttt{Dense(1)} mit Sigmoid-Aktivierung und explizitem \texttt{float32}-Output, um trotz Mixed Precision stabile und exakte Ausgaben zu gewährleisten.
\end{itemize}

Im finalen Modell ist der gesamte ResNet50-Block von Beginn an trainierbar (\texttt{base\_model.trainable = True}). Auf ein stufenweises Fine-Tuning einzelner Layer wird somit verzichtet; alle Schichten werden gleichberechtigt im Gradientenabstieg aktualisiert.

\section{Vorverarbeitung und Datenpipeline}

Die Bildvorverarbeitung und das Datenhandling erfolgen vollautomatisch durch den \texttt{ImageDataGenerator} aus \texttt{tensorflow.keras.preprocessing.image}. Als Preprocessing wird die \texttt{preprocess\_input}-Funktion von ResNet50 verwendet, die neben der Skalierung auf $256\times256$~Pixel auch eine kanalweise Normalisierung vornimmt. 

Im Gegensatz zu frühen Projektphasen wird im finalen Ansatz komplett auf künstliche Datenaugmentation (Rotation, Flip, Zoom etc.) verzichtet, da empirische Tests gezeigt haben, dass diese im Gesamtsystem keine Vorteile bringen, sondern die Stabilität sogar vermindern können.

Bilder werden batchweise (Batchgröße 64) geladen und die Labels (\texttt{real} / \texttt{fake}) automatisch aus der Ordnerstruktur als binäre Klassen extrahiert.

\section{Trainingskonfiguration und Hyperparameter}

Die Modellkompilierung erfolgt mit dem Adam-Optimierer (Standardparameter), \emph{binary\_crossentropy} als Verlustfunktion und \emph{accuracy} als zentrale Trainingsmetrik. Wesentliche Trainingsparameter:

\begin{itemize}
    \item Input-Größe: $256\times256\times3$
    \item Batchgröße: 64
    \item Trainingsdauer: 50~Epochen (mit früherem Abbruch durch Callback möglich)
    \item Mixed Precision Policy: \texttt{mixed\_float16}
\end{itemize}

\textbf{Callbacks} für Monitoring und Steuerung:

\begin{itemize}
    \item \texttt{ModelCheckpoint}: Speichert das jeweils beste Modell (nach Validierungs-Accuracy, Datei: \texttt{best\_model\_initial.h5}).
    \item \texttt{ReduceLROnPlateau}: Reduziert die Lernrate bei Stagnation der Validierungsverluste (Faktor 0{,}2, Patience 10, min\_lr $1\mathrm{e}{-7}$).
    \item \texttt{TensorBoard}: Logging aller Trainingsmetriken und Lernverläufe zur späteren Analyse.
\end{itemize}

\section{Ablauf des Trainings und der Evaluation}

Der vollständige Trainingsprozess lässt sich wie folgt beschreiben:

\begin{enumerate}
    \item Initialisierung des Modells und Laden der Trainings- und Validierungsdaten über die ImageDataGenerator-Pipeline.
    \item Training über bis zu 50~Epochen mit Echtzeit-Monitoring aller Metriken (Trainings-/Validierungs-Accuracy und -Loss).
    \item Speichern des besten Modells während des Trainings per Callback.
    \item Nach Abschluss des Trainings erfolgt die Evaluation auf dem separaten Test-Set:
        \begin{itemize}
            \item Berechnung von \textbf{Loss} und \textbf{Accuracy} mittels \texttt{model.evaluate(...)}
            \item Erstellung eines detaillierten \textbf{Klassifikationsreports} (Precision, Recall, F1-Score) mittels \texttt{sklearn.metrics.classification\_report}
            \item Darstellung der \textbf{Konfusionsmatrix} mittels \texttt{sklearn.metrics.confusion\_matrix}
        \end{itemize}
    \item Das finale Modell wird als \texttt{model\_final.h5} gespeichert.
\end{enumerate}

\section{Versuchsplanung und Entwicklungsschritte}

Im Verlauf des Projekts wurden zahlreiche Alternativen und Erweiterungen getestet, darunter:

\begin{itemize}
    \item \textbf{Datenaugmentation:}  
        Zu Beginn wurden diverse Augmentations (Rotation, Flip, Zoom) zur Erhöhung der Varianz erprobt. Sie wurden jedoch im finalen Ansatz entfernt, da sie zu schwankenden und weniger stabilen Trainingsergebnissen führten.
    \item \textbf{Stufenweises Fine-Tuning:}  
        Ein schrittweises Training mit initial eingefrorenem Feature-Extractor und nachfolgend sukzessiv freigegebenen Layern (z.\,B. 10/50/100 Schichten) brachte keinen konsistenten Vorteil gegenüber dem sofortigen End-to-End-Training.
    \item \textbf{Hyperparameter-Tuning:}  
        Verschiedene Lernraten ($\eta=10^{-4}$ bis $10^{-6}$), Regularisierungsstärken (L2, Dropout), Batchgrößen und Optimierer wurden evaluiert. Die finale Konfiguration orientiert sich an den stabilsten und am besten generalisierenden Settings.
    \item \textbf{Architekturvergleich:}  
        Alternative Backbones wie EfficientNetB0 wurden verglichen, letztlich fiel die Entscheidung jedoch klar zugunsten von ResNet50 aus.
    \item \textbf{Datensatzdiversifizierung:}  
        Das Einbinden zusätzlicher Datensätze (wie „Detect AI-Generated Faces“) führte zu einer Erhöhung der Varianz und geringfügigen Verbesserungen im Recall, jedoch nicht zu signifikant besseren Gesamtergebnissen.
\end{itemize}

\section{Zusammenfassung}

Die beschriebene Methodik gewährleistet ein reproduzierbares, transparentes und effizientes Training eines Deep-Learning-Modells zur Deepfake-Erkennung, wobei sämtliche Entscheidungen datengestützt und iterativ empirisch begründet wurden.


% ─────────────────────────────────────────────────────────────────────────────
% 4. Implementierung    Momentan unangebracht da keine Codeabschnitte nötig
% ─────────────────────────────────────────────────────────────────────────────
% %─────────────────────────────────────────────────────────────────────────────
% Kapitel 4: Implementierung
%─────────────────────────────────────────────────────────────────────────────
\chapter{Implementierung}
\label{chap:implementierung}

Im Folgenden skizzieren wir die wichtigsten Code-Snippets und die Projektstruktur des Trainingsskripts. Die komplette Code‐Basis befindet sich im Verzeichnis \texttt{src/} (nicht Bestandteil dieser Arbeit). 

\section{Projektstruktur des Trainingsskripts}
\begin{verbatim}
Projekt/
|-- data/
|   |-- train/
|   \-- test/
|-- models/
|   |-- model1.h5
|   \-- model2.h5
\-- src/
    |-- train.py
    \-- evaluate.py
\end{verbatim}


\section{Wichtige Codeausschnitte}
\subsection{Initialisierung des Modells}
\begin{lstlisting}[language=Python, caption=ResNet50-Modell-Definition aus \texttt{train\_model.py}]
import tensorflow as tf
from tensorflow.keras.applications import ResNet50
from tensorflow.keras.layers import GlobalAveragePooling2D, Dense, Dropout
from tensorflow.keras.regularizers import l2

def create_model(freeze_base: bool = True, l2_factor: float = 0.002):
    base_model = ResNet50(weights='imagenet', include_top=False, input_shape=(256,256,3))
    base_model.trainable = not freeze_base
    
    x = base_model.output
    x = GlobalAveragePooling2D()(x)
    x = Dense(128, activation='relu', kernel_regularizer=l2(l2_factor))(x)
    x = Dropout(0.6)(x)
    output = Dense(1, activation='sigmoid')(x)
    
    model = tf.keras.Model(inputs=base_model.input, outputs=output)
    return model
\end{lstlisting}

\subsection{Training und Fine-Tuning}
\begin{lstlisting}[language=Python, caption=Training‐Loop aus \texttt{train\_model.py}]
# Phase 1: Initialtraining
model = create_model(freeze_base=True)
model.compile(
    optimizer=tf.keras.optimizers.Adam(learning_rate=1e-4),
    loss='binary_crossentropy',
    metrics=['accuracy', tf.keras.metrics.AUC()]
)

early_stop = tf.keras.callbacks.EarlyStopping(monitor='val_loss', patience=10, restore_best_weights=True)
checkpoint_init = tf.keras.callbacks.ModelCheckpoint(
    'best_model_initial.h5', monitor='val_accuracy', save_best_only=True)

tensorboard_cb = tf.keras.callbacks.TensorBoard(log_dir='logs/fit/initial', update_freq='epoch')

model.fit(
    train_dataset,
    epochs=20,
    validation_data=val_dataset,
    callbacks=[early_stop, checkpoint_init, tensorboard_cb]
)

# Phase 2: Fine-Tuning
for layer in model.layers[-50:]:
    layer.trainable = True

model.compile(
    optimizer=tf.keras.optimizers.Adam(learning_rate=1e-6),
    loss='binary_crossentropy',
    metrics=['accuracy', tf.keras.metrics.AUC()]
)

checkpoint_ft = tf.keras.callbacks.ModelCheckpoint(
    'best_model_finetuned.h5', monitor='val_accuracy', save_best_only=True)
reduce_lr = tf.keras.callbacks.ReduceLROnPlateau(
    monitor='val_loss', factor=0.2, patience=2, min_lr=1e-7)

model.fit(
    train_dataset,
    epochs=20,
    validation_data=val_dataset,
    callbacks=[early_stop, checkpoint_ft, reduce_lr, tensorboard_cb]
)
\end{lstlisting}


% ─────────────────────────────────────────────────────────────────────────────
% 5. Evaluation
% ─────────────────────────────────────────────────────────────────────────────
%─────────────────────────────────────────────────────────────────────────────
% Kapitel 5: Evaluation
%─────────────────────────────────────────────────────────────────────────────
\chapter{Evaluation}
\label{chap:evaluation}

Nach dem Training wurden die Modelle auf dem Testset (ca. 10 \% der Daten) evaluiert. Es wurden folgende Metriken ausgewertet:
\begin{itemize}
  \item \textbf{Accuracy:} Anteil korrekt klassifizierter Bilder.  
  \item \textbf{Precision / Recall / F1-Score:} Aus dem \texttt{sklearn.metrics}–Klassifikationsreport.  
  \item \textbf{AUC:} Area Under the ROC-Kurve (verwaltet durch Keras während des Trainings).
  \IfFileExists{figures/confusion_matrix.png}{%
    \item \textbf{Konfusionsmatrix:} Visualisierung mit \texttt{matplotlib} (siehe Abbildung \ref{fig:confusion}).
  }{%
    \item \textbf{Konfusionsmatrix:} {\color{red}Bild nicht gefunden, deshalb hier ausgelassen.}
  }
    
\end{itemize}

\section{Quantitative Ergebnisse}
\begin{table}[h]
\centering
\caption{Test-Ergebnisse verschiedener Modellvarianten}
\label{tab:results}
\begin{tabular}{lcccccc}
\toprule
Modell                    & Acc. & Prec. & Rec. & F1-Score & AUC  & Fine-Tune-Layer\\
\midrule
best\_initial (20/0)      & 0.7030 & 0.7866 & 0.5785 & 0.6667 & –     & – \\
best\_finetuned (20/20-50)& 0.6914 & 0.7762 & 0.5608 & 0.6511 & –     & 50 \\
\bottomrule
\end{tabular}
\end{table}

\section{Konfusionsmatrix}
\IfFileExists{figures/confusion_matrix.png}{%
  \begin{figure}[h]
    \centering
    \includegraphics[width=0.6\textwidth]{figures/confusion_matrix.png}
    \caption{Konfusionsmatrix des finalen Modells auf dem Testset}
    \label{fig:confusion}
  \end{figure}
}{%
  \begin{center}
   {\color{red}\textbf{Warnung:} Bild nicht gefunden: \texttt{figures/confusion\_matrix.png}\\
  Die Konfusionsmatrix wurde in dieser Version weggelassen.}
  \end{center}
}

\section{Analyse der Ergebnisse}
Die Ergebnisse zeigen, dass das Fine-Tuning mit zu vielen freigegebenen Layern (z. B. 50) zu Overfitting führte, während ein vorsichtiger Feinschliff (10 Layer) die beste Mischung aus Precision und Recall erreichte. Die AUC-Kurve (Abbildung \ref{fig:roc}) bestätigt, dass das Modell selbst bei ungleichen Klassenverteilungen noch einigermaßen trennscharf bleibt.

\IfFileExists{figures/roc_curve.png}{%
  \item \textbf{ROC-Kurve:} Auswertung der True-/False-Positive-Rate (siehe Abbildung \ref{fig:roc}).
}{%
  \item {\color{red}\textbf{ROC-Kurve:} Bild nicht gefunden, deshalb hier ausgelassen.}
}



% ─────────────────────────────────────────────────────────────────────────────
% 6. Diskussion
% ─────────────────────────────────────────────────────────────────────────────
%─────────────────────────────────────────────────────────────────────────────
% Kapitel 6: Diskussion
%─────────────────────────────────────────────────────────────────────────────
\chapter{Diskussion}
\label{chap:diskussion}

Obwohl das initiale Training mit eingefrorenem ResNet50 breite Generalisierung lieferte (Accuracy ~ 0,70), zeigt sich in den Feintuning-Durchläufen, dass zu umfangreiches Fine-Tuning (50 Layer) zu einer Reduktion der Recall-Rate führt (0,50 gegenüber 0,61). Mögliche Ursachen:
\begin{itemize}
  \item \textbf{Datenheterogenität:} Unterschiedliche Bildquellen (verschiedener Kaggle‐Datensätze) weisen nicht dieselben Kameraeigenschaften auf. Das Feintuning kann lokale Artefakte überanpassen.  
  \item \textbf{Overfitting:} Trotz starker L2-Regularisierung und erhöhtem Dropout (0,6) zeigten die TensorBoard‐Kurven nach Epochen 15–20 einen deutlichen Divergenz‐Effekt zwischen Training und Validierung (vgl. Anhang, Abbildung A.1).  
  \item \textbf{Fehlende große Testdaten:} Die finale Test-Stichprobe umfasste nur ca. 10 % der Bilder. Eine größere, externe Testmenge (z. B. von selbst gesammelten Realbildern) könnte die Aussagekraft verbessern.
\end{itemize}

\section{Limitationen der aktuellen Arbeit}
\begin{itemize}
  \item Systematisch wurden weder \emph{Video-Deepfakes} noch Audio-Manipulationen berücksichtigt.  
  \item Die Datensätze enthalten primär Gesichter in neutraler Mimik – extreme Gesichtsausdrücke (Lachen, Grimassen) wurden kaum repräsentiert.  
  \item Die Trainingsdauer (je Lauf 4–6 Stunden auf einer RTX 2080 Ti) begrenzte das Experimentieren mit noch tieferen Netzwerken (z. B. EfficientNetV2-Large).  
\end{itemize}


% ─────────────────────────────────────────────────────────────────────────────
% 7. Fazit
% ─────────────────────────────────────────────────────────────────────────────
%─────────────────────────────────────────────────────────────────────────────
% Kapitel 7: Fazit
%─────────────────────────────────────────────────────────────────────────────
\chapter{Fazit}
\label{chap:fazit}

Diese Projektarbeit zeigt, dass ein auf ResNet50 basierendes Transfer-Learning‐Ansatz in der Lage ist, Deepfake‐Bilder von echten Aufnahmen mit einer Accuracy von ca. 0,70 zu unterscheiden. Durch gezieltes Fine-Tuning (Freigabe nur der letzten 10 Layer) konnte ein guter Kompromiss zwischen Precision und Recall gefunden werden. Die Integration von L2‐Regularisierung und erhöhtem Dropout verhindert zumindest teilweise Overfitting. 

Für zukünftige Arbeiten wäre eine Ausweitung auf Video-Deepfakes sinnvoll, ebenso wie die Integration von multimodalen Ansätzen (z. B. Audio+Bild) und die Erforschung von Ensemble-Methoden.  

Insgesamt leistet diese Arbeit einen Beitrag zur automatisierten Erkennung von Deepfakes und kann als Grundlage für weiterführende Projekte dienen.  


% ─────────────────────────────────────────────────────────────────────────────
% Literaturverzeichnis
% ─────────────────────────────────────────────────────────────────────────────
\clearpage
\printbibliography[title={Literaturverzeichnis}]

% ─────────────────────────────────────────────────────────────────────────────
% Anhang
% ─────────────────────────────────────────────────────────────────────────────
\clearpage
%─────────────────────────────────────────────────────────────────────────────
% Anhang: Code, weitere Tabellen, TensorBoard‐Plots etc.
%─────────────────────────────────────────────────────────────────────────────
\chapter{Anhang}
\label{chap:anhang}

\section{Detailierte Versuche (Auszug)}
Im Folgenden ein Auszug aus den Versuchsdokumenten (vgl. \texttt{versuche.txt}):

\begin{itemize}
  \item ResNet50 durch EfficientNetB0 ersetzt  
  \item Functional API genutzt statt \texttt{Sequential()}  
  \item Mixed Precision aktiviert (float16)  
  \item Fehler: \enquote{Unable to serialize … JSON. Unrecognized type <EagerTensor>}  
  \item … (weitere Notizen siehe Datei \texttt{versuche.txt} im Projektordner)
\end{itemize}

\section{Zusätzliche Plots}
\begin{figure}[h]
    \centering
    \includegraphics[width=0.7\textwidth]{figures/tensorboard_training_loss.png}
    \caption{Trainingsverlauf (Loss) aus TensorBoard}
    \label{fig:tboard_loss}
\end{figure}

\section{Training‐Dokumentation (Auszug)}
Auszug aus \texttt{train_model_doku.txt} (siehe Projektordner):
\begin{quote}
“Training eines CNN-Modells zur binären Klassifikation (z.B. ‚Echt‘ vs. ‚KI-generiert‘) mit Hilfe von Transfer Learning (ResNet50) unter Verwendung von TensorFlow / Keras. …”  
\end{quote}
n

\end{document}
%─────────────────────────────────────────────────────────────────────────────
